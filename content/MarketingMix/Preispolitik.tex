\subsection{Preisstrategie} \label{preisstrat}
    Bei der Wahl unserer Preisstrategie haben wir uns für die Premium-Strategie entschieden:
    
        \begin{itemize}
            \item Qualität
            
                Da wir in unserem Produkt eine hohe Qualität bieten, sind wir der Ansicht das wir für diese Produkt auch
                einen höher segmentierten Preis verlangen können.

            \item Image
            
                Durch ein stetig hohe Qualität wollen wir uns ein gutes Image aufbauen, in welchem unser Unternehmen für
                gute Qualität steht.

            \item Hohe Kosten

                Der Bau eines Roboters ist mit vielen Kosten verbunden. Ein sehr wichtiges Bauteil, das für alle Roboter
                benötigt wird, sind Chips. Doch in aktuellen Zeiten ist Chip-Industrie am Kapazitätslimit und die Preise
                sind hoch. Um diese Kosten stämmen zu können, müssen wir das Produkt im Premium Segment absetzen

            \item Made in Germany
            
                Unsere Produktion findet vollständig hier in Deutschland statt. Dadurch das wir hier produzieren, sind
                auch unsere Produktionskosten höher, was sich in unserem Preis wiederspiegelt. Allerdings steht \as Made
                in Germany\adl auch für eine gute Qualität und auch für gut Arbeitsbegingungen der Angestellten. Das
                hilft uns auch dabei ein gutes Image aufzubauen.

            \item Kundendienst
            
                Mit dem Kundendienst bieten wir eine zusätzliche Leistung über den eigentlichen Kauf des Produktes
                an. Damit diese zusätzlichen Kosten gedeckt werden können, muss der Preis auf einem höheren Niveau 
                angesetzt werden.

            \item Nischenmarkt
            
                Da es sich bei unserem Produkt um einen Nischenmarkt handelt, sind möglicherweise viele Menschen nicht 
                bereit das Geld für den \as RinnenRobo\adl auszugeben. Außerdem ist dies ein nicht sehr bekannter Markt.
                Um dieses Defizit kompensieren zu können haben wir uns für die Premium-Strategie entschieden.
        \end{itemize}

\subsection{Preisdifferenzierung} \label{Preisdiff}
    \begin{itemize}
        \item Zeitliche Strategie
            
        Bei der Zeitlichen Strategie haben wir uns für die Skimming-Strategie entschieden. Der Grund dafür ist, 
        wir grade zur Markteinführung einen höheren Absatz erwarten. 

        Ein weiterer Grund für diese Strategie ist, dass es sich um ein technisches Produkt handelt. In dieser
        Sparte wollen die meisten Menschen immer das neueste Produkt. Das bedeutet das sie nach einiger Zeit 
        nicht mehr bereit sind unseren Roboter zu einem sehr hohen Preis zu erwerben.

        \item Kundenbezogen

            Unser Unternehmen hat einen Unterschiedlichen Preis für Kunden aus dem B2B und dem B2C Bereich. Der Grund 
            hierfür liegt in unseren verschiedenen Produkten, welche wir einmal speziel für den B2C und den B2B Bereich
            Designed haben.

        \item Mengenbezogen
        
            Bei Bestellungen von großen Mengen bieten wir einen Mengenrabatt an. Wenn sich ein 
            Dienstleistungsunternehmen eine ganze Flotte Roboter kauft, hat dies für uns Kostentechnische Vorteile.
            Wir müssen das System bei dem Kunden nur einmal Einrichten und Erklären. Das bedeutet das wir bei einer 
            Bestellung von zum Beispiel 10 Robotern einen Ähnlichen Aufwand haben wie bei einer Bestellung von nur einem 
            Roboter.
        \end{itemize}