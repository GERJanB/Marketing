\section{Produktpolitik}
    Der Begriff Produktpolitik umfasst alle Entscheidungen, die sich auf die Gestaltung des Angebotes eines Unternehmen
    beziehen. Die Produktpolitik ist neben der Preispolitik, der Kommunikationspolitik und der Distributionspolitik ein
    Element im Marketing-Mix. Sie beschäftigt sich mit der Gestaltung des Produktes (Eigenschaften, Name, Verpackung),
    des Service (Beratung, Wartung) und des Sortiments (Angebote, Produktvariation). (Vgl. \cite{Bruhn2012}, S.\,28-29)

    \subsection{Ziele der Produktpolitik}
        Die Produktpolitik erfolgt zielorientiert, wobei verschiedene Ziele angestrebt werden.

        Ökonomische Ziele der Produktpolitik sind das Erzielen eines Gewinns, die Sicherstellung des Wachstums, die
        Optimierung der Wettbewerbsposition, die Minimierung des Risikos, eine möglichst hohe Kapazitätsauslastung sowie
        die Realisierung von Rationalisierungs- und Synergieeffekten.

        Psychologische Ziele der Produktpolitik ist die Steigerung des Goodwills (beispielsweise Firmenimage,
        Technologieführerschaft) sowie die positive Beeinflussung der Einstellung, mit der die Kaufinteressenten das
        Produkt beurteilen. (Vgl. \cite{Bruhn2012}, S.\,130)

    \subsection{Nutzenkomponenten eines Produkts}
        Ein Produkt ist ein Bündel von nutzenstiftenden Eigenschaften, dass die Befriedigung von Kundenbedürfnissen zum
        Ziel hat. Es wird zwischen zwei Nutzungsarten unterschieden.

        Grundnutzen: Mit dem Grundnutzen werden die grundlegenden Bedürfnisse eines Menschen befriedigt. Die Unternehmen
        nutzen den Grundnutzen, um Produkte herzustellen, die auf die Grundbedürfnisse abgestellt sind. Zu den
        Grundbedürfnissen zählen z. B. Wärme, Hunger und Mobilität.

        Zusatznutzen: Mit dem Zusatznutzen erfüllt das Produkt eines Unternehmens einen Nutzen, der über den Grundnutzen
        hinausgeht.

            \begin{itemize}
                \item Funktionaler Zusatznutzen
                
                    Über den Grundnutzen hinausgehende Bedürfnisbefriedigung durch das Produkt.

                \item Emotionaler Zusatznutzen
                
                    Aus den Emotionalen Wirkungen eines Produktes resultierende Bedürfnisbefriedigung.

                \item Sozialer Zusatznutzen
                
                    Aus den Sozialen Wirkungen eines Produktes resultierende Bedürfnisbefriedigung.
            \end{itemize}

        Im Fall \as RinnenRobo\adl hat das Produkt das Grundnutzen der Reinigung von Dachrinnen. Als Funktionaler
        Zusatznutzen kann zu dem Produkt eine Wetterstation installiert werden. Zudem wird die Inspektion des
        \as RinnenRobo\adl kostenfrei angeboten. 
            
        Der Emotionale Zusatznutzen gibt den Produktanwendern, in diesem Fall oftmals Personen der älteren Generation,
        ein Gefühl der Selbständigkeit, da sie unabhängig sind und keine persönliche Hilfe von anderen Personen erbitten
        müssen. 
        
        Der Soziale Zusatznutzen besteht darin, dass Anwender dieses Produktes Technisches Interesse zeigen. 
            
    \subsection{Produktvarianten}
        Ein Produkt kann in verschiedenen Varianten desselben Produkts verkauft werden, zum Beispiel in verschiedenen
        Farben und Größen. Wenn ein Produkt Varianten hat, heißt das übergeordnete Produkt, zu dem die Varianten
        erstellt werden, Hauptprodukt. Jede Variante kann einen eigenen Preis, eigene Bestandsinformationen,
        Produktinformationen usw. haben (Vgl. \cite{Bruhn2012}, S.\,131, 157).
        
        Das Produkt \as RinnenRobo\adl gibt es in zwei Hauptvarianten, für den privaten Anwender in drei Ausführungen
        und für die Unternehmer und Dienstleister in zwei Ausführungen.
            
        \subsubsection{RinnenRobo}
            \begin{itemize}
                \item RinnenRobo Light
                
                    Diese Variante ist günstiger als die beiden anderen. Sie hat nur eine Hauptaufgabe und zwar das
                    Reinigen der Dachrinne und keine zusätzlichen Features. 

                \item RinnenRobo Pro
                
                    Neben der Hauptaufgabe hat die Pro Variante alle zusätzlichen Features. Bei der Pro Variante kann
                    eine Wetterstation mit gebucht werden. Sie kann aktuelle Wetterdaten wiedergeben und mit dem
                    Smartphone können alle Funktionen abgerufen werden. Da sie alle Features enthält ist sie teuerste
                    Variante in der RinnenRobo Produktgruppe. 

                \item RinnenRobo Modular
                
                    Diese Variante hat unterschiedliche Preise, da beliebige Features neben der Hauptfunktion dazu
                    gewählt werden können. 
            \end{itemize}

        \subsubsection{RinnenRobo Business}
            \begin{itemize}
                \item RinnenRobo Rental
                
                    Da dieses Produkt teuer in der Anschaffung ist wird es von Dienstleistern erworben und diese bieten
                    es zum Mieten an.  

                \item RinnenRobo Smart
                
                    Diese Variante ist ähnlich wie die Pro Variante, ist aber für Unternehmen gedacht. Sie hat alle
                    Features und zusätzlich eine feste Ladestation. Die Unternehmen haben hier auch noch die Möglichkeit
                    ihr eigenes Firmenlogo auf den RinnenRobo aufzuzeichnen. 
            \end{itemize}