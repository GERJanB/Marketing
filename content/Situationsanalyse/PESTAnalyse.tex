\subsection{PEST-Analyse} \label{pest}
    Bei der PEST-Analyse bzw. der PEST(EL)-Methode um eine Überischtliche Darstellung der Makro-Umwelt des Unternehmens
    zu bekommen. Sie wurde im Jahr 1986 von Fahey und Narayanan entwickelt. Dabei werden die wesentlichen Rahmenfaktoren
    und Trends der globalen Unternehmensumwelt erfasst und 
    bewertet. Bei der Bewertung wird geschaut, welche Auswirkung die Faktoren der Makro-Umwelt auf das eigene 
    Unternehmen haben. Im folgenden werden einmal die einzelnen Teile der PEST-Methode erklärt und danach auf unser 
    Unternehmen bezogen.

        \begin{itemize}
            \item \textbf Political (Politisch)

                Hierbei handelt es sich die politische Situation des Landes in welchem das Unternehmen seinen Sitz hat 
                bzw. in welchem Land das Unternehmen produzieren oder verkaufen möchte. Dabei wird auf die Gesetzgebung 
                und auf die politischen Regulierungen geschaut und welche Auswirkungen diese auf zum Beispiel den Absatz
                des Unternehmens haben. Dazu zählt auch ob es für das geplante Produkt Regulierungen bezüglich der 
                legalität oder der 

            \item \textbf Economical (Wirtschaftlich)
            \item \textbf Socio-Demographic (Sozio-Demographisch)
            \item Technological (Technologisch)
        \end{itemize}
