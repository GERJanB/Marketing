\subsection{Kommunikationspolitik} \label
    Nachdem nun die Produktpolitik betrachtet wurde, folgt die Kommunikationspolitik. Dieses Werkzeug des 
    Marketing-Mixes beschreibt die Wege, mit denen das Unternehmen mit seinen Kunden in Kontakt tritt und diese mit
    Informationen und Werbung versorgt. (Vgl. \cite{Kuss2016} S.\,203)

    \noindent
    Bei der Entwicklung der Kommunikationspolitik ist die Auswahl der Marketingträger und -plattformen von essenzieller
    Bedeutung. Um die wirkungsvollsten Träger auszuwählen, müssen zunächst wieder die Personas aus der Situationsanalyse
    betrachtet werden. Danach werden für diese Zielgruppen entsprechende Kommunikationswege gewählt.

    \noindent Im Fall \as RinnenRobo\adl gibt es die folgeden zwei Zielgruppen:

    \begin{enumerate}
        \item Hauseigentümer, vor allem ältere Personen
        \item Dienstleister in der Gebäudepflege
    \end{enumerate}

    \noindent Um für diese Zielgruppen nun die Kommunikationsmedien zu wählen, wird mit der Intermediaselektion begonnen.
    
    \noindent Für die erste Zielgruppe wurden folgende Medien herausgestellt:

        \begin{itemize}
            \item Fernsehen
            \item Radio
            \item Zeitung
        \end{itemize}

    \noindent Um die Dienstleister zu erreichen, wurden die folgenden Medien gewählt: 

        \begin{itemize}
            \item Fachmessen
            \item Fachmagazine
            \item Direktmarketing
        \end{itemize}

    \noindent
    Bei der Auswahl dieser Medien wurde betrachtet, auf welchen Medien die Personen der entsprechenden Zielgruppe häufig
    anzutreffen sind. Somit wären hier auch die Chancen für eine erfolgreiche Marketingkampagne hoch.

    \noindent
    Im nächsten Schritt werden die einzelnen Kommunikationskanäle genauer betrachtet und es findet die
    Intramediaselektion statt. Hierbei werden konkrete Möglichkeiten innerhalb der Plattformen herausgearbeitet. Dabei
    wird wieder versucht, die Teilkanäle zu wählen, bei denen ein Antreffen von potenziellen Kunden am
    wahrscheinlichsten ist.

    \noindent
    Die Intramediaselektion für die Gruppe der Hauseigentümer sieht folgendermaßen aus:

    \begin{itemize}
        \item Fernsehen
            \subitem Das Erste
            \subitem ZDF
            \subitem NDR
        \item Radio
            \subitem NDR 1
            \subitem Bremen 1
        \item Zeitung
            \subitem Lokale Tageszeitung
    \end{itemize}

    \noindent
    Für die Dienstleister wurden folgende Teilkanäle gewählt:

     \begin{itemize}
        \item Messen
            \subitem IPM Essen
            \subitem GaLaBau
            \subitem CMS Messe Berlin
        
        \item Fachmagazine
            \subitem Der Hausmeister
            \subitem Galabau Journal

        \item Direktmarketing
            \subitem Prospekte
            \subitem Flyer
     \end{itemize}

    \noindent
    Auf den genannten Teilkanälen wäre also eine Kommunikation mit Personen der entsprechenden Zielgruppe möglich.

    \noindent
    Um nun konkrete Marketingmaßnahmen planen und durchführen zu können, muss außerdem eine Budgetstrategie festgelegt
    werden. Diese Strategie besagt, wie viel Kapital für die Kommunikationspolitik ausgegeben werden darf. Dieses
    Kapital wird beispielsweise für Fernsehspots oder Werbeanzeigen in der lokalen Tageszeitung benötigt.
    (Vgl. \cite{Bruhn2014a}, S.\,212)

    \noindent
    Es gibt sowohl analytische als auch heuristische Verfahren zur Budgetierung. Bei den analytischen Ansätzen wird das
    verfügbare Kapital durch mathematische Funktionen errechnet, während dieses bei heuristischen Ansätzen nach
    vereinfachten Regeln festgelegt wird (vgl. \cite{Bruhn2014a}, S.\,214).

    \noindent
    Im Folgenden werden jedoch nur die heuristischen Ansätze weiter betrachtet.

    \noindent
    Diese heuristischen Verfahren lassen sich in drei weitere Untergruppen gliedern. Diese werden im Folgenden genannt
    und näher erläutert.

    \begin{itemize}
        \item Unternehmensbezogene Ansätze
            
            Bei dieser Art von Budgetierung werden unternehmensinterne Werte zur Budgetkalkulation herangezogen. So kann
            beispielsweise ein gewisser Prozentsatz des Umsatzes für Kommunikationsmaßnahmen eingesetzt werden.

        \item Konkurrenzbezogene Ansätze
        
            Hier orientiert sich ein Unternehmen an den Werbebudget und -ausgaben der Konkurrenz und versucht auf dessen
            Grundlage die eigene Marketingbudgetierung vorzunehmen.

        \item Marktbezogene Ansätze
        
            Bei diesen Ansätzen liegt der Fokus auf dem zu erreichenden Ziel. Daraus wird das benötigte Budget bestimmt.
            So kann das Ziel etwa eine Bekanntheit von 20\% auf dem deutschen Markt sein. Hieraus wird nun das zum
            Erreichen dieses Ziels benötigte Budget bestimmt.
    \end{itemize} (Vgl. \cite{Bruhn2014a}, S.\,214)

    \noindent
    Im Praxisbeispiel \as RinnenRobo\adl wird ein unternehmensbezogener Ansatz genutzt. Genauer: Es soll so viel
    Kapital für die Durchführung der Marketingmaßnahmen genutzt werden, wie verfügbar ist.

    \noindent
    Diese Entscheidung ist begründet durch das allgemeine Unwissen über die Existenz von Rinnenreinigungsrobotern und
    den Wunsch \as RinnenRobo\adl zu einer bekannten Marke zu entwickeln, um auch höhere Preise rechtfertigen zu
    können.

