\subsection{Marketingziele}
    Marketingziele werden formuliert, um die Ressourcen in einem Betreib gemeinsam in eine Richtung einzusetzen. Dabei
    geben die Ziele bei Entscheidungen eine Orientierung und sollen die Mitarbeiter motivieren. Die Ziele müssen
    erreichbar sein, damit die Mitarbeiter nicht demotiviert werden. Ein wichtiger Aspekt der Ziele ist die
    anschließende Kontrolle der Zielerfüllung und Abweichungsanalyse, um ggf. neue Maßnahmen ableiten zu können. Diese
    Ziele sind mittel- bis langfristig ausgelegt. (Vgl. \cite{Becker2018}, S.\,60)

    \noindent
    Um die unterschiedlichen Aspekte und Vorgaben der Zielsetzung zu berücksichtigen kann die SMART-Methode angewandt
    werden Die SMART-Methode geht auf Peter Drucker zurück und bietet Kriterien zur eindeutigen Formulierung von
    überprüfbaren und umsetzbaren Zielen. Jeder Buchstabe von SMART beinhaltet ein Kriterium an die Zielformulierung.
    Werden alle Kriterien eingehalten, können Unternehmensziele formuliert werden. (Vgl. \cite{Lawlor2012}, S.\,269)

    \begin{itemize}
        \item Spezifisch
        
            Das Ziel muss eindeutig und konkret formuliert sein. Es dürfen keine Missverständnisse entstehen. Bsp.: 
            \as Der Umsatz ist in diesem Jahr gegenüber dem Vorjahr zu steigern.\ad

        \item Messbar

            Die Erreichung des Ziels muss messbar sein. So werden quantitative und qualitative Anforderungen an das Ziel
            gesetzt. Eine objektive Bewertung muss möglich sein. Bsp.: \as Der Umsatz ist in diesem Jahr gegenüber dem
            Vorjahr um 15\% zu steigern.\ad

        \item Attraktiv
        
            Die Zielsetzung muss für die Teammitglieder attraktiv und akzeptabel sein. Können die Mitglieder sich nicht
            mit dem Ziel identifizieren so sinkt die Motivation. So sollte z.B. eine gemeinnützige Unternehmung keine
            moralisch verwerflichen Ziele anstreben. 

        \item Realistisch
        
            Das Ziel muss realistisch und damit umsetzbar formuliert sein. Ist die Erreichung des Ziels unrealistisch
            sinkt die Motivation das Ziel zu erreichen. Werden Zwischenziele erreicht kann das die Motivation weiter
            antreiben.

        \item Terminiert
        
            Um das Erreichen der Ziele kontrollieren zu können muss ein Zeitpunkt zur Zielerfüllung festgelegt sein.
            Folglich ist das Ziel in einer bestimmten Zeit zu erreichen. Nachdem die Zeit abgelaufen ist, kann die
            Kontrolle und Abweichungsanalyse starten.
    \end{itemize}


    %TODO Section machen

    \noindent
    Die Marketingziele für den RinnenRobo

    \begin{enumerate}
        \item Ziel
        
            Der RinnenRobo soll innerhalb eines Jahres 25\% Bekanntheit auf dem deutschen Fachmarkt erreicht haben.

        \item Ziel
        
            Nach drei Jahren sollen 10.000 Einheiten auf dem deutschen Massenmarkt (Privat- und Geschäftskunden)
            abgesetzt worden sein.

        \item Ziel
        
            Auf dem europäischen Markt soll eine Bekanntheit von ca. 5\% erreicht werden. Das bedeutet jeder zwanzigste
            Europäer soll Berührungspunkte mit unserem Produkt gesammelt haben. Dazu zählt auch Werbung.
    \end{enumerate}
