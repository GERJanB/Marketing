\subsection{Persona A} \label{personaA}
    Die erste Persona, welche wir uns überlegt haben, ist Clemens Müller. Clemens ist 70 Jahre alt. Er ist verheiratet 
    und hat Kinder, welche allerdings weit entfernt wohnen. Er hat in seinem Leben als Ingenieur gearbeitet, ist 
    allerdings mittlerweile berentet. Seine hauptsächlichen Informationsquellen sind die Zeitung und das Fernsehen. In 
    seiner Freizeit ist er Heimwerker oder bastelt an Eletronischen Geräten. 

    Er ein Interesse an unserem Produkt, da er sich eine körperliche Entlastung wünscht und selbstständiger sein möchte.
    In unserem Produkt such eine leichte Bedienbarkeit, eine lange Haltbarkeit und einen guten Service.

\subsection{Persona B} \label{personaB}
    Bei unserer zweiten Persona handelt es sich um Andreas Martens. Andreas ist 40 Jahre alt, verheiratet und hat zwei 
    Kinder. Er interressiert sich für unser Produkt, da er es sehr gut für seine Arbeit in der Gebäudepflege, welche er 
    seit 20 Jahren ausübt, einsetzen kann. Außerdem ist er sehr Technikbegeistert und ist auf der Suche nach 
    alternativen Arbeitswegen, mit welchen er sich seine Arbeit erleichtern und die Sicherheit erhöhen kann. Ebenfalls 
    ist er auf der Suche nach mehr Effizienz und nach einem Zuverlässigen Produkt.

    Über neue technologische Entwicklungen informiert er sich vor allem über Fachmagazine, wie zum Beispiel \as Der
    Hausmeister\adl, oder auch über das Radio.