\subsection{Distributionspolitik} \label{distro}
    Distribution beschreibt die betriebliche Funktion zwischen Hersteller und Verbraucher. Damit beschreibt die
    Distributionspolitik alle Entscheidungen und Maßnahmen auf dem Weg vom Anbieter zum Konsumenten.

    \noindent
    Die Distributionspolitik ist ein maßgebendes Instrument wie an den Kunden herangetreten wird. Die
    Distributionsentscheidungen sind im Regelfall langfristig ausgelegt. Es müssen je nach Produkt komplexe
    Distributionsketten auf- und ausgebaut werden.

    \noindent
    Die Distributionspolitik verfolgt dabei Ökonomische-, Versorgungs- und Psychologische Ziele. Hauptbestandteil der
    Ökonomischen Ziele ist der Erhalt und Ausbau des Betriebs in dem z.B. neue Kundengruppen erschlossen werden. Die
    Versorgungsziele beschäftigen sich unter anderem mit Lieferzuverlässigkeit und Liefergeschwindigkeit. Der dritte
    Bereich, die Psychologischen Ziele beinhalten insbesondere das Auftreten und die Wahrnehmung des Unternehmens.
    