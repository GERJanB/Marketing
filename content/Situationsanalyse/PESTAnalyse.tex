\subsection{PEST-Analyse} \label{pest}
    Bei der PEST-Analyse bzw. der PEST(EL)-Methode um eine Überischtliche Darstellung der Makro-Umwelt des Unternehmens
    zu bekommen. Sie wurde im Jahr 1986 von Fahey und Narayanan entwickelt. Dabei werden die wesentlichen Rahmenfaktoren
    und Trends der globalen Unternehmensumwelt erfasst und 
    bewertet. Bei der Bewertung wird geschaut, welche Auswirkung die Faktoren der Makro-Umwelt auf das eigene 
    Unternehmen haben. Im folgenden werden einmal die einzelnen Teile der PEST-Methode erklärt und danach auf unser 
    Unternehmen bezogen.

        \begin{itemize}
            \item \textbf Political (Politisch)

                Hierbei handelt es sich die politische Situation des Landes in welchem das Unternehmen seinen Sitz hat 
                bzw. in welchem Land das Unternehmen produzieren oder verkaufen möchte. Dabei wird auf die Gesetzgebung 
                und auf die politischen Regulierungen geschaut und welche Auswirkungen diese auf zum Beispiel den Absatz
                des Unternehmens haben. Dazu zählt auch ob es für das geplante Produkt Regulierungen bezüglich der 
                legalität oder Absatzmengen einschränkungen gibt.

                Desweiteren wird die politische Stabilität des Landes in betracht gezogen. Denn auch diese kann 
                Auswirkungen auf die Entwicklung des Unternehmens haben. Bei einer Instabilen Lage kann es dazu kommen,
                dass sich das Unternehmen nicht am Markt halten kann und im schlimmsten Fall Insolvenz anmelden muss.

            \item \textbf Economical (Wirtschaftlich)
            
                Neben der politischen Lage ist auch die Wirtschaftliche Lage sehr wichtig für das Unternehmen. In diesem
                Teil der Analyse wird vorallem darauf geachtet, wie sich der Markt in der Region des Unternehmens bzw. 
                am Absatzort entwickelt. Es muss festgestellt werden ob ein gesättigter oder ungesättigter Markt 
                vorliegt und ob es für das Unternehmen möglich ist im gewünschten Markt Fuß zu fassen. Ein weiterer 
                wichtiger Faktor aus diesem Bereich sind die Steuerlichen Abgaben. Je nach höhe können diese ein 
                negativen Einfluss auf die Entwicklung des Unternehmens haben.

            \item \textbf Socio-Demographic (Sozio-Demographisch)
            
                Dieser Teil der Analyse befasst sich mit der Demographischen Situation. Wichtige Faktoren in diesem 
                Bereich sind zum Beispiel Trends oder das Kosumverhalten der Einwohner des Absatzlandes. Wenn zum 
                Beispiel die gewünschten Produktgruppe im gewünschten Absatzland nicht sehr beliebt ist und wenig 
                verkauft wird, sollte dies zu großen Teilen mit in die Entscheidung, ob an diesem Ort verkauft wird, 
                einfließen. Auch Ethnische und religöse Faktoren können hier in Betracht gezogen werden, wenn diese 
                Berührungspunkte mit dem abzusetzenden Produkt haben.

            \item \textbf Technological (Technologisch)
            
                Grade bei einem Unternehmen welches oft gute technologische Standards und deren Entwicklung angewiesen 
                ist, ist dieser Faktor sehr wichtig. Wenn ein Unternehmen zum Beispiel sehr viel über das Internet 
                arbeitet, also Online-Shops, Marketing etc., sollte sich dieses Unternehmen einen Standort aussuchen,
                welcher die nötige Infrastruktur bietet um das Konzept umzusetzen.

                Bei einem Unternehmen welche viel an der Forschung neuer Technologien arbeiten, wäre es hilfreich für 
                das Unternehmen wenn Forschung am möglichen Standort gefördert wird.
        \end{itemize}

    Beim Übertragen dieser Punkte auf das eigene Unternehmen sind wir zu folgenden Bewertungen unserers Standortes hier
    in Deutschland gekommen:

        \begin{itemize}
            \item Political
            
                Deutschland hat eine Politisch stabile Lage und gesetztlich gibt es wenige Regulierung. Auch der Export
                und Import sind durch die Europäische Union leicht möglich.

                Ein Thema welches allerdings in Deutschland immer mehr an Ansehen gewinnt, ist der Umweltschutz. Dies 
                führt dazu das wir bei unserer Produktion möglichst Umweltschonend vorgehen sollten, um das Ansehen des
                Unternehmens hier zu stärken und unserer Ziele  zu erreichen.

            \item Economical
            
                Deutschland hat eine insgesamt gute wirtschaftliche Situation. Durch aktuelle Ereignisse in der Welt 
                haben sich die Wechselkurse im Vergleich zu den Vorjahren verschlechter. Außerdem sind die steuerlichen 
                Abgaben hier in Deutschland vergleichweise hoch, was negative Auswirkungen auf unser Betriebsergebnis
                haben kann.

                Der Markt für einen Reinigungsroboter für Dachrinnen ist in Deutschland sehr klein, da es nur ein 
                anderes Konkurrzenprodukt gibt.

            \item Socio-Demographic
            
                In Deutschland gibt es eine zunehmend alternde Bevölkerung was unserem Unternehmen zugute kommt, da eine
                unserer beiden Zielgruppen, die älteren Personen sind, welche selbst nicht mehr in der Lage sind ihre 
                eigenen Dachrinnen sauber zu machen und Verwandte nicht in der Nähe sind um dies für sie zu machen.

                Außerdem gibt es hier in Deutschland eine hohe Affinität im Bereich der Technik und verschiedner Gadgets
                was unserem Unternehmen ebenfalls zugute kommt. Auch lassen sich viele Kosumenten mittlerweile viel mehr 
                zu Spontankäufen verleiten, was grade im Bereich der Technik sehr verbreitet ist.

            \item Technological
            
                Deutschland bietet gute technologische Entwicklung und auch hohe Ausgaben im Bereich der Forschung. 
                Aufgrund der hohen Preise für menschliche Arbeit, müssen wir einen großen Teil unserer Produktion 
                automatisieren um Kosten einzusparen.

                Ein großer Nachteil an Deutschland ist in diesem Bereich allerdings die Vernetzung, da grade der Ausbau
                des Breitbandnetzes hier nur schleppend vorran kommt.
        \end{itemize}