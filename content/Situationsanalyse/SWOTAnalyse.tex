\subsection{SWOT-Analyse} \label{swot}
Um nun externe und interne Informationsbereiche miteinander zu verbinden, wird die \as SWOT-Analyse\adl genutzt. Mit
dieser Betrachtung ist es möglich, die Stärken und Schwächen des Unternehmens kombiniert mit den Chancen und Risiken der
Mikro- und Makroumwelt zu untersuchen und Strategien abzuleiten. (Vgl. \cite{Brun2014} S.\,43)

\noindent Dabei wurden zum Praxisbeispiel „RinnenRobo“ die folgenden Chancen erarbeitet:

    \begin{itemize}
        \item \textbf Demographischer Wandel
        
            Da das Produkt \as RinnenRobo\adl den Menschen das Säubern ihrer Dachrinnen erleichtert und auch sicherer
            gestaltet, ist es besonders für ältere Menschen interessant. Diese können möglicherweise auf Grund von
            körperlichen Einschränkungen nicht mehr per Hand die Dachrinnen reinigen. Somit vergrößert der
            demographische Wandel die Gruppe der potenziellen Abnehmer in diesem Bereich.
        \item \textbf Technologische Trends
            
            Das Produkt \as RinnenRobo\adl könnte außerdem großes Interesse bei technisch-interessierten Menschen
            wecken. Auch die Größe dieser Gruppe könnte durch den Trend zur Digitalisierung und Technologisierung
            wachsen.
        \item \textbf Förderung von Innovation
        
            Da es nur ein Produkt gibt, welches in ähnlicher Weise zum \as RinnenRobo\adl funktioniert, wäre es denkbar,
            eine finanzielle Förderung zur Entwicklung des Roboters einzustreichen. Diese Förderung könnte dabei
            entweder staatlicher oder privater Natur sein.
    \end{itemize}

\noindent Nachdem nun die Chancen betrachtet wurden, sollen im Folgenden die Risiken näher betrachtet werden:

    \begin{itemize}
        \item Weniger Eigenheime

            In Deutschland gibt es momentan einen sinkenden Anteil an Eigenheimen und einen dadurch bedingten, höheren
            Anteil an Mietwohnungen und -häusern (Vgl. \cite{Mueller2021}). Dieser Umstand könnte dazu führen, dass weniger
            \as RinnenRobos\adl verkauft werden, da davon ausgegangen werden kann, dass Privatpersonen mit Eigenheim
            mehr Wert auf dessen Gepflegtheit legen als Personen ohne Eigenheim.

        \item Steigende Produktionskosten
        
            Durch höhere Rohstoff- und Logistikkosten werden die Produktionskosten momentan stark erhöht. Diese
            Preiserhöhung muss im Endverkaufspreis an den Kunden weitergegeben werden. Auf Grund dieser Preissteigerung
            könnten sich einige potenzielle Käufer doch gegen den Erwerb eines \as RinnenRobos\adl entscheiden.
    \end{itemize}

\noindent Nachdem der externe Informationsbereich nun betrachtet wurde, ist nachfolgend der interne Informationsbereich
zu untersuchen. Begonnen wird mit den Stärken des Unternehmens:

    \begin{itemize}
        \item Arbeitserleichterung und Zeitersparnis
        
            Durch das Produkt \as RinnenRobos\adl sparen sich Käufer viel Zeit bei der Reinigung ihrer Dachrinnen. Außerdem ist
            die Arbeit auch körperlich stark erleichtert. Diese Vorteile könnte das Unternehmen zur Gewinnung neuer
            Kunden nutzen.

        \item Erhöhte Sicherheit
        
            Auch führt die Benutzung des \as RinnenRobos\adl zu einer höheren Sicherheit beim Säubern der Dachrinnen. Auch
            dies ist ein weiteres Verkaufsargument für das Unternehmen.
    \end{itemize}